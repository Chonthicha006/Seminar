\documentclass[a4paper]{article}
\usepackage{amsmath}
\usepackage{amsfonts}
\usepackage{amssymb}
\usepackage{graphicx}
\usepackage{amsthm}

\usepackage[text={7in,10in},centering]{geometry}
\usepackage{fancyhdr}
\usepackage[no-math]{fontspec}
\usepackage{polyglossia}
\pagenumbering{gobble}
\usepackage{verbatim}
\immediate\write18{texcount -tex -sum  \jobname.tex > \jobname.wordcount.tex}
% Keywords command
\providecommand{\keywords}[1]
{
  \large 	
  \textbf{\textit{Keywords---}} #1
}
\fancyhead{สสสส}
\title{{\textbf{สมการไดโอเเฟนไทน์ {\large{$2^{x}+p^{y}=z^{2}$}} เมื่อ {\large{$x\neq1$}} และ {\large{$ p\equiv 3\left ( 3mod4 \right )$}}\\
โดย สุธน ตาดี}}}
\author{\Large{\textbf{นำเสนอโดย} นางสาวชลธิชา พ่วงเฟื่อง}\\
        \Large\textbf{อาจารย์ที่ปรึกษา} :  ผศ. ดร.ทรงพล ศรีวงค์ษา\\}
\date{} % Comment this line to show today's date
\setmainfont{TH Sarabun New}
\begin{document}
\maketitle

\begin{center}\Large\textbf{บทคัดย่อ}
\end{center}
\vspace{5pt}
\hspace*{30pt}\Large       {งานวิจัยนี้เป็นการศึกษาเกี่ยวกับสมการไดโอแฟนไทน์(Diophantine equation) ซึ่งสมการไดโอเฟนไทน์เป็นสมการที่ได้ศึกษากัน\\อย่างมากมายและศึกษาในหลายรูปแบบ รูปแบบหนึ่งที่ได้ศึกษากันอย่างกว้างขวางคือ 
สมการที่อยู่ในรูป   $a^{x}+b^{y}=z^{2}$ \\ในงานวิจัยนี้เกี่ยวกับการหาผลเฉลยของสมการไดโอแฟนไทน์ 
โดยดังกล่าวได้มีการนำเสนอทฤษฎีบทและแนวคิดบางประการที่เกี่ยวข้องกับ\\
สมการไดโอแฟนไทน์และนำข้อความคาดการณ์กาตาลัน(Catalan’s conjecture) มาใช้ประโยชน์ในการพิสูจน์ครั้งนี้โดยแยกกรณี \\และหาผลเฉลยทั้งหมดที่เป็นจำนวนเต็ม
ที่ไม่เป็นลบของสมการไดโอแฟนไทน์ $2^{x}+p^{y}=z^{2}$ เมื่อ $x\neq1$ และ $p\equiv 3\left ( 3mod4 \right )$
จะแสดงให้เห็นว่า ผลเฉลยที่เป็นจำนวนเต็มที่ไม่เป็นลบของสมการไดโอแฟนไทน์นี้อยู่ในรูปแบบต่อไปนี้เท่านั้น คือ\\
$\left ( x,p,y,z \right )\in \left \{ \left (3,p,0,3  \right ) \right \}\cup \left \{ \left ( 0,3,1,2 \right ) \right \}\cup \left \{ \left ( 2+log _{2}\left ( p+1 \right )\right ),p,2,p+2):log_{2}\left ( p+1 \right )      \in   \mathbb{Z}    \right \}$}
\hspace{30pt}
\hspace{30pt}
\newpage %ขึ้นหน้าใหม่



%TC:ignore
%\keywords{one, two, three, four}
\begin{thebibliography}{00}
\bibitem{b2} Clarke, Arthur C. 2001: A Space Odyssey. New York: Roc, 1968. 297.
\end{thebibliography}
%TC:endignore

% Word count
\verbatiminput{\jobname.wordcount.tex}

\end{document}